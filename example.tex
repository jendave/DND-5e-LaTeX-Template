\documentclass[letterpaper,twocolumn,openany,nodeprecatedcode]{dndbook}

% Use babel or polyglossia to automatically redefine macros for terms
% Armor Class, Level, etc...
% Default output is in English; captions are located in lib/dndstrings.sty.
% If no captions exist for a language, English will be used.
%1. To load a language with babel:
%	\usepackage[<lang>]{babel}
%2. To load a language with polyglossia:
%	\usepackage{polyglossia}
%	\setdefaultlanguage{<lang>}
\usepackage[english]{babel}
%\usepackage[italian]{babel}
% For further options (multilanguage documents, hypenations, language environments...)
% please refer to babel/polyglossia's documentation.

\usepackage[utf8]{inputenc}
\usepackage[singlelinecheck=false]{caption}
\usepackage{lipsum}
\usepackage{listings}
\usepackage{shortvrb}
\usepackage{xtab}
\usepackage[open,openlevel=1]{bookmark}
\captionsetup[table]{labelformat=empty,font={sf,sc,bf,},skip=0pt}

% Uncomment to allow the default background paper image to be overridden.
%\SetPaperImage{img/paper}

\MakeShortVerb{|}

\lstset{%
  basicstyle=\ttfamily,
  language=[LaTeX]{TeX},
  breaklines=true,
}

\title{The Dark \LaTeX{}\\ \large An Example of the dndbook Class}
\author{The rpgTeX Team}
\date{2020/04/21}

\hypersetup{
  pdfauthor={David Hudson},%
  pdftitle={Kami no Kuni},%
  pdfsubject={RPG Sourcebook},%
  pdfkeywords={RPG, 5E, Sourcebook},%
  pdfproducer={LaTeX},%
  pdfcreator={LuaLaTeX},%
  hidelinks
}

\makeindex[intoc,name=index,title=Index]
\indexsetup{  
  level=\chapter*,%
  toclevel=chapter,%
}

\addbibresource{references.bib}

\renewcommand{\DndTitle}{The Dark LaTeX\\ \fontsize{30pt}{30pt}\selectfont An Example of the dndbook Class}
\renewcommand{\DndCoverSplotchText}{HOMEBREW}
\renewcommand{\DndTagline}{LaTeX template for the world's greatest roleplaying game}
\renewcommand{\DndCoverPageBackground}{img/cover}

\begin{document}

\frontmatter

% \maketitle
\DndMakeCover%
\DndBlankPage%
\DndMakeSubcover%

% \tableofcontents

\setcounter{tocdepth}{1}
\pdfbookmark{\contentsname}{toc}
{
	\hypersetup{hidelinks}
	\tableofcontents
}

\mainmatter%

\include{example-part-1.tex}
\part{Customization}

\chapter{Colors}

\begin{table*}[b]
  \caption{\DndFontTableTitle{}Colors Supported by this Package}\label{tab:colors}

  \begin{tabularx}{\linewidth}{lX}
    \textbf{Color}                  & \textbf{Description} \\
    \rowcolor{PhbLightGreen}
    |PhbLightGreen|                 & Light green used in PHB Part 1 (Default) \\
    \rowcolor{PhbLightCyan}
    |PhbLightCyan|                  & Light cyan used in PHB Part 2 \\
    \rowcolor{PhbMauve}
    |PhbMauve|                      & Pale purple used in PHB Part 3 \\
    \rowcolor{PhbTan}
    |PhbTan|                        & Light brown used in PHB appendix \\
    \rowcolor{DmgLavender}
    |DmgLavender|                   & Pale purple used in DMG Part 1 \\
    \rowcolor{DmgCoral}
    |DmgCoral|                      & Orange-pink used in DMG Part 2 \\
    \rowcolor{DmgSlateGray}
    |DmgSlateGray| (|DmgSlateGrey|) & Blue-gray used in PHB Part 3 \\
    \rowcolor{DmgLilac}
    |DmgLilac|                      & Purple-gray used in DMG appendix \\
    \rowcolor{BrGreen}
    |BrGreen|                       & Gray-green used for tables in Basic Rules\\
  \end{tabularx}
\end{table*}

This package provides several global color variables to style |DndComment|, |DndReadAloud|, |DndSidebar|, and |DndTable| environments.

\begin{DndTable}[header=Box Colors]{lX}
  \textbf{Color}   & \textbf{Description} \\
  |commentcolor|   & |DndComment| background \\
  |readaloudcolor| & |DndReadAloud| background \\
  |sidebarcolor|   & |DndSidebar| background \\
  |tablecolor|     & background of even |DndTable| rows \\
\end{DndTable}

They also accept an optional color argument to set the color for a single instance. See Table~\ref{tab:colors} for a list of core book accent colors.

\begin{lstlisting}
\begin{DndTable}[color=PhbLightCyan]{cX}
  \textbf{d8} & \textbf{Item} \\
  1 & Small wooden button \\
  2 & Red feather \\
  3 & Human tooth \\
  4 & Vial of green liquid \\
  6 & Tasty biscuit \\
  7 & Broken axe handle \\
  8 & Tarnished silver locket \\
\end{DndTable}
\end{lstlisting}

\begin{DndTable}[color=PhbLightCyan]{cX}
  \textbf{d8} & \textbf{Item} \\
  1 & Small wooden button \\
  2 & Red feather \\
  3 & Human tooth \\
  4 & Vial of green liquid \\
  6 & Tasty biscuit \\
  7 & Broken axe handle \\
  8 & Tarnished silver locket \\
\end{DndTable}

\section{Themed Colors}
Use |\DndSetThemeColor[<color>]| to set |commentcolor|, |readaloudcolor|, |sidebarcolor|, and |tablecolor| to a specific color. Calling |\DndSetThemeColor| without an argument sets those colors to the current |themecolor|. In the following example the group limits the change to just a few boxes; after the group finishes, the colors are reverted to what they were before the group started.

\begin{lstlisting}
\begingroup
\DndSetThemeColor[PhbMauve]

\begin{DndComment}{This Comment Is in Mauve}
  This comment is in the the new color.
\end{DndComment}

\begin{DndSidebar}{This Sidebar Is Also Mauve}
  The sidebar is also using the new theme color.
\end{DndSidebar}
\endgroup
\end{lstlisting}

\begingroup
\DndSetThemeColor[PhbMauve]

\begin{DndComment}{This Comment Is in Mauve}
  This comment is in the the new color.
\end{DndComment}

\begin{DndSidebar}{This Sidebar Is Also Mauve}
  The sidebar is also using the new theme color.
\end{DndSidebar}
\endgroup


\backmatter%

\chapter{Appendix A: Inspiration}

Here is an example of a bibliography listing sources that inspired this work.
\nocite{*}
\printbibliography[heading=subbibintoc,type=book,title={Books}]
\printbibliography[heading=subbibintoc,keyword={movie},title={Movies}]
\printbibliography[heading=subbibintoc,keyword={game},title={Games}]
\printbibliography[heading=subbibintoc,keyword={website},title={Websites}]



{\hypersetup{hidelinks}
	\printindex[index]
}

\renewcommand*{\DndBackcover}{img/backcover-image}
\renewcommand{\DndBackcoverHeader}{LaTeX Template}
\renewcommand{\DndBackcoverClose}{This template provides the a layout for the fifth-edition rulebooks.}
\renewcommand{\DndBackcoverDescription}{This template provides the a layout for the fifth-edition rulebooks. It uses LuaLatex and contains code by from rpgtex, ashonit, jendave and Artefact2.}
\renewcommand{\DndBackcoverLink}{\small\url{https://github.com/jendave/DND-5e-LaTeX-Template}}
\renewcommand*{\DndBackcoverLogo}{img/cover-logo-homebrewery}
\DndMakeBackcover%

\end{document}
